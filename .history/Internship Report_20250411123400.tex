\documentclass[a4paper,12pt]{report}

\usepackage{geometry}
\geometry{margin=1in}
\usepackage{graphicx}
\usepackage{amsmath}
\usepackage{hyperref}
\usepackage{listings}
\usepackage{xcolor}
\lstset{%
    basicstyle=\ttfamily\small,
    frame=single,
    backgroundcolor=\color{gray!10},
    keywordstyle=\color{blue},
    commentstyle=\color{green!50!black},
    stringstyle=\color{orange},
    numbers=left,
    numberstyle=\tiny,
    numbersep=5pt,
    showstringspaces=false,
    breaklines=true
}

\begin{document}

\begin{titlepage}
    \centering
    \vspace*{2cm}
    {\Huge\bfseries Internship Report\\[0.3cm]}
    \vspace{1cm}
    {\LARGE Analysis and Evaluation of the Python Library: Chromatix}\\[1.5cm]
    {\large Your Name}\\[0.2cm]
    {\large Internship Period: XX-XX-XXXX to XX-XX-XXXX}\\[0.2cm]
    {\large Supervisor: Supervisor Name}\\[2cm]
    \includegraphics[width=0.4\textwidth]{university_logo.png}\\[2cm]
    \vfill
\end{titlepage}

\tableofcontents
\newpage

\chapter{Introduction}
\section{Internship Objective}
Brief description of your internship goal and tasks assigned.

\section{Overview of Chromatix}
Introduction to the Chromatix library, its purpose, features, and general applications in optical systems.

\chapter{Methodology}
\section{Analytical Approach}
Description of the methods used to analyze the library (e.g., code review, documentation review, theoretical background analysis).

\section{Tools and Resources}
List and describe the resources (documentation, software tools) used during the analysis.

\chapter{Analysis of Chromatix}
\section{Main Classes and Functions}
Detailed analysis of the main classes and functions within the library, including theoretical background and practical code explanations.

\subsection{Class 1: Name}
Detailed explanation with theoretical background and code snippet examples.

\subsection{Class 2: Name}
Detailed explanation with theoretical background and code snippet examples.

% Add additional classes/functions as subsections as needed.

\section{Evaluation}
Critical evaluation of the strengths, weaknesses, and limitations of the Chromatix library.

\chapter{Application Examples}
\section{Example 1: Building Optical System A}
Description, purpose, implementation, and results.

\section{Example 2: Building Optical System B}
Description, purpose, implementation, and results.

\chapter{Conclusion and Future Work}
\section{Conclusion}
Summary of your findings, overall impressions, and effectiveness of Chromatix.

\section{Future Work}
Recommendations for further study or improvements to the library and potential applications.

\bibliographystyle{plain}
\bibliography{references}

\appendix
\chapter{Additional Material}
Include additional scripts, code snippets, or supplementary material here.

\end{document}
