\documentclass[a4paper,12pt]{report}

\usepackage{geometry}
\geometry{margin=1in}
\usepackage{graphicx}
\usepackage{amsmath}
\usepackage{hyperref}
\usepackage{listings}
\usepackage{xcolor}
\usepackage{tabularx}

\lstset{%
    basicstyle=\ttfamily\small,
    frame=single,
    backgroundcolor=\color{gray!10},
    keywordstyle=\color{blue},
    commentstyle=\color{green!50!black},
    stringstyle=\color{orange},
    numbers=left,
    numberstyle=\tiny,
    numbersep=5pt,
    showstringspaces=false,
    breaklines=true
}

\begin{document}

\begin{titlepage}
    \begin{minipage}[t]{0.5\textwidth}
        \includegraphics[width=1\textwidth]{ASP.jpeg}
    \end{minipage}
    \hfill
    \begin{minipage}[t]{0.5\textwidth}
        \raggedleft
        \includegraphics[width=1\textwidth]{IAP.jpg}
    \end{minipage}


    \vspace{1cm}

    {\Huge\centering\bfseries Internship Report\\[0.5cm]}
    \vspace{1cm}

    \raggedright
    \large

    {\centering\bfseries Topic: Report on Chromatix\\}

    \vspace{2cm}

    \centering\textbf{Author:}\\[0.5cm]
    \centering Ziyi Xiong\\[0.5cm]
    \centering Master Student in Photonics at ASP\\[0.5cm]
    
    \centering\textbf{Internship Period:}\\[0.5cm] 
    \centering  01-09-2024 to 28-02-2025\\[0.5cm]
    
    \centering \textbf{Supervisor:} \\[0.5cm]
    \centering Prof. Dr. Vladan Blahnik, Institute of Applied Physics\\[0.5cm]
    
    \centering\textbf{Industry Advisor:} \\[0.5cm]
    \centering Prof. Dr. Frank Wyrowski, President of LightTrans GmbH\\

    \vfill
\end{titlepage}


\tableofcontents
\newpage

\chapter{Introduction}
\section{Context of the Internship}
The internship was conducted under the joint guidance of my university supervisor, Prof. Dr. Vladan Blahnik, and my industry advisor, Prof. Dr. Frank Wyrowski. The project entailed a deep-dive analysis into the open-source Python library known as \textbf{Chromatix}, which was originally developed at HHMI Janelia Research Campus. This library leverages wave-optics and advanced computational frameworks (notably JAX and FLAX) to enable \textit{differentiable} simulations of optical systems.\newline
\newline
My role was to thoroughly understand Chromatix, from its underlying wave-optical theory to the structure of its Python source code, and to explore real use cases. Specifically, I examined how classes, functions, and modules contribute to building optical elements, performing wave propagation, and enabling inverse-design tasks such as optimizing lens parameters and masks.

\section{Goals and Objectives}
The primary goals of this internship were:
\begin{enumerate}
    \item \textbf{Comprehensive Code Analysis:} Understand and document the organization and functionality of the Chromatix library (classes, functions, modules).
    \item \textbf{Theoretical Foundations:} Relate each function's or class's implementation to the fundamental physics of wave optics (e.g., Fresnel diffraction, Fourier transforms, Jones calculus, etc.).
    \item \textbf{Practical Demonstrations:} Demonstrate usage through multiple example systems (simple widefield PSF simulation, a 4f system, and a CGH setup using a Digital Micromirror Device (DMD)).
    \item \textbf{Evaluation \& Future Prospects:} Critically assess strengths, limitations, and propose future directions for library development.
\end{enumerate}

\section{Structure of This Report}
This report is divided into the following chapters:
\begin{itemize}
    \item \textbf{Chapter 1: Introduction} - Provides an overview of the internship objectives and a brief introduction to Chromatix. Delves into the design principles, module structure, and theoretical basis of the library.
    \item \textbf{Chapter 2: Core Source Code Analysis} - Explores the main classes and functions in Chromatix, mapping them to wave-optical principles.
    \item \textbf{Chapter 3: Practical Usage Examples} - Demonstrates real-world simulations using Chromatix with extended code snippets.
    \item \textbf{Chapter 4: Conclusions \& Future Work} - Summarizes key findings and offers recommendations for further improvements.
\end{itemize}

\chapter{Chromatix in Detail}
\section{Overview}
Chromatix is a differentiable wave-optics library designed to model complex optical systems while allowing gradient-based optimization, thanks to the JAX ecosystem. Key features include:
\begin{itemize}
    \item \textbf{GPU Acceleration}: Offloading large-scale computations to GPUs or TPUs.
    \item \textbf{Inverse Design}: Using auto-differentiation to iteratively refine system parameters (e.g., lens curvature, phase masks).
    \item \textbf{Modular Architecture}: Dividing code into multiple packages (\texttt{data}, \texttt{functional}, \texttt{elements}, etc.) for clarity and maintainability.
    \item \textbf{Multi-Wavelength \& Polarization Handling}: Scalar and vector fields, and spectral weighting.
\end{itemize}

\section{Package Breakdown}
\textbf{Chromatix} is organized into six major packages plus two additional modules:
\begin{itemize}
    \item \textbf{data}: Generates fundamental optical patterns (radial shapes, grayscale images) and can create permittivity tensors for materials. While not central to wave-propagation, it provides ready-made data for simulation.
    \item \textbf{functional}: Implements physical transformations like Fresnel or Angular Spectrum propagation, lens-induced phase changes, amplitude/phase masks, polarizers, and sensor operations. This package is the mathematical backbone, reliant heavily on JAX (\texttt{jax.numpy}).
    \item \textbf{elements}: Encapsulates optical components as trainable modules (subclasses of \texttt{flax.linen.Module}). For example, \textit{lens elements}, \textit{phase mask elements}, etc. This design allows each element to be optimized via gradient-based methods.
    \item \textbf{systems}: Provides the \texttt{OpticalSystem} class to combine multiple \textit{elements} into a pipeline. Example systems include a standard microscope (\texttt{Microscopy}) and a 4f system (\texttt{Optical4FSystemPSF}).
    \item \textbf{ops}: Supplies typical image-processing utilities (filters, noise, quantization) that can be integrated into pipeline simulations.
    \item \textbf{utils}: Hosts a variety of helper functions (FFT wrappers, 2D/3D shape generation, Zernike polynomials, wavefront aberrations, etc.) to support the core simulation tasks.
    \item \textbf{field.py}: Defines the \texttt{Field} class (with \texttt{ScalarField} and \texttt{VectorField}) that serve as the data structure for complex wave fields. Includes sampling intervals, wavelength arrays, and advanced indexing.
    \item \textbf{\_\_init\_\_\.py}: Basic initializations, ensuring that the library is well-structured upon import.
\end{itemize}

\section{Core Theoretical Concepts}
Chromatix leverages fundamental wave optics:
\begin{enumerate}
    \item \textbf{Fourier Optics}: The library heavily uses Fourier transforms to handle propagations (Fresnel, Fraunhofer, Angular Spectrum).
    \item \textbf{Jones Calculus}: Particularly for vector fields, polarizers, and wave plates.
    \item \textbf{Sampling Theory}: Ensuring that the discretized fields respect sampling requirements, particularly for FFT-based methods.
    \item \textbf{Differentiable Programming}: JAX's automatic differentiation engine allows every wave transform to be differentiable w.r.t. design parameters.
\end{enumerate}

\chapter{Core Source Code Analysis}
\section{Field Class (\texttt{field.py})}
The \textbf{Field} class (and its scalar/vector variants) is the centerpiece:
\begin{itemize}
    \item \textbf{u}: A complex array representing the electromagnetic field samples. Typically shaped as $(B..., H, W, C, [1|3])$, where $H\times W$ is the spatial grid, $C$ is the number of spectral channels, and $[1|3]$ indicates scalar or vector fields.
    \item \textbf{\_dx}: Sampling intervals (y, x) specifying real-space pixel sizes.
    \item \textbf{\_spectrum}: One or more wavelengths used in the field (multi-wavelength simulations supported).
    \item \textbf{\_spectral\_density}: Respective weights of each wavelength.
\end{itemize}
\noindent \textbf{Key Methods} include:
\begin{itemize}
    \item \texttt{grid()}: Returns spatial coordinate arrays, centered in the middle of the array.
    \item \texttt{k\_grid()}: Returns frequency coordinates (Fourier domain) suitable for propagation computations.
    \item \texttt{phase()}: Extracts the phase of the complex field.
    \item \texttt{intensity()}: Computes the intensity $I=|u|^2$.
    \item \texttt{amplitude()}: Computes the amplitude $|u|$.
    \item Arithmetic operators (\texttt{\_\_add\_\_}, \texttt{\_\_mul\_\_}, etc.) that permit easy combinations of fields.
\end{itemize}

\noindent\textbf{Physical Rationale}: By bundling sampling intervals and spectral data within \texttt{Field}, Chromatix simplifies transform-based operations (Fourier transforms, lens phases, etc.). The methods ensure consistent coordinate handling.

\subsection{Why JAX?}
JAX transforms each function into a computation graph that can be differentiated. For wave optics, each transformation (e.g., lens phase or Fresnel propagation) is re-cast in a way that JAX can keep track of partial derivatives. This is extremely useful for tasks such as inverse design, where we want to adjust lens surfaces or mask patterns to achieve a target field.

\section{Functional - Module \& Function Overview}
The \textbf{functional} package provides static (pure) functions implementing standard optical transformations:
% in your preamble:
\usepackage{enumitem}

% … later in the document …
\section{Functional – Module \& Function Overview}

\begin{itemize}[itemsep=1em]  % space between each file‐block
  \item[\large\bfseries 1.\ ] amplitude\_masks.py
    \begin{itemize}[itemsep=0.5em]  % space between functions
      \item \texttt{amplitude\_change}: This function perturbs a Field object by a given amplitude array.
    \end{itemize}

  \item[\large\bfseries 2.\ ] convenience.py
    \begin{itemize}[itemsep=0.5em]
      \item \texttt{optical\_fft}: This \texttt{optical\_fft} function performs a Fresnel propagation on a Field object using a Fast Fourier Transform (FFT) or Inverse Fast Fourier Transform (IFFT) depending on the propagation direction.
    \end{itemize}

  \item[\large\bfseries 3.\ ] lenses.py
    \begin{itemize}[itemsep=0.5em]
      \item \texttt{thin\_lens}: Calculates the field after passing a thin lens.
      \item \texttt{ff\_lens}: Simulates a 4f‐style system: propagate distance \(f\)→thin lens→propagate distance \(f\).
      \item \texttt{df\_lens}: General 2f setup: initial distance \(d\)→thin lens→propagate \(f\).
    \end{itemize}

  \item[\large\bfseries 4.\ ] phase\_masks.py
    \begin{itemize}[itemsep=0.5em]
      \item \texttt{spectrally\_modulate\_phase}: Models multi‐wavelength phase modulation.
      \item \texttt{phase\_change}: Perturbs the input field by a given phase array.
      \item \texttt{wrap\_phase}: Wraps phase values into a specified range.
    \end{itemize}

  \item[\large\bfseries 5.\ ] polarizers.py
    \begin{itemize}[itemsep=0.5em]
      \item \texttt{jones\_vector}: Creates a Jones vector from angle \(\theta\) and delay \(\beta\).
      \item \texttt{linear}: Jones vector for linearly polarized light.
      \item \texttt{left\_circular}: Jones vector for left‐circular polarization.
      \item \texttt{right\_circular}: Jones vector for right‐circular polarization.
      \item \texttt{polarizer}: General polarizer (linear/circular/custom).
      \item \texttt{linear\_polarizer}: Applies a linear polarizer at a given angle.
      \item \texttt{left\_circular\_polarizer}: Applies a left‐circular polarizer.
      \item \texttt{right\_circular\_polarizer}: Applies a right‐circular polarizer.
      \item \texttt{phase\_retarder}: Applies an arbitrary phase‐retarder Jones matrix.
      \item \texttt{wave\_plate}: Shorthand for a linear phase retarder (wave plate).
      \item \texttt{halfwave\_plate}: Introduces a \(\pi\) phase shift between orthogonal components.
      \item \texttt{quarterwave\_plate}: Introduces a \(\pi/2\) phase shift.
      \item \texttt{universal\_compensator}: Generates any desired polarization state.
    \end{itemize}

  \item[\large\bfseries 6.\ ] propagation.py
    \begin{itemize}[itemsep=0.5em]
      \item \texttt{transform\_propagate}: Single‑FFT Fresnel propagation (SFT‑FR).
      \item \texttt{compute\_sas\_precompensation}: Precompensation for Scaled Angular Spectrum (SAS).
      \item \texttt{kernel\_propagate}: Propagate via Fourier‐domain convolution.
      \item \texttt{transform\_propagate\_sas}: SAS propagation using precompensation.
      \item \texttt{compute\_transfer\_propagator}: Builds the transfer‐function kernel for Fresnel.
      \item \texttt{transfer\_propagate}: Computes propagator + applies \texttt{kernel\_propagate}.
      \item \texttt{compute\_exact\_propagator}: Kernel for exact (no‑approx) propagation.
      \item \texttt{exact\_propagate}: Exact propagation via kernel + convolution.
      \item \texttt{compute\_asm\_propagator}: Angular‐spectrum kernel (with output‐plane shift).
      \item \texttt{asm\_propagate}: Angular‐spectrum propagation over \(z\).
      \item \texttt{compute\_padding\_transform}: Compute required padding for transform method.
      \item \texttt{compute\_padding\_transfer}: Compute required padding for transfer method.
      \item \texttt{compute\_padding\_exact}: Compute required padding for exact method.
    \end{itemize}

  \item[\large\bfseries 7.\ ] pupils.py
    \begin{itemize}[itemsep=0.5em]
      \item \texttt{circular\_pupil}: Applies a circular aperture mask.
      \item \texttt{square\_pupil}: Applies a square aperture mask.
    \end{itemize}

  \item[\large\bfseries 8.\ ] samples.py
    \begin{itemize}[itemsep=0.5em]
      \item \texttt{jones\_sample}: Vector‑field perturbation by a thin sample.
      \item \texttt{thin\_sample}: Scalar‑field thin‑sample perturbation.
      \item \texttt{multislice\_thick\_sample}: Thick‐sample via multiple thin slices.
      \item \texttt{fluorescent\_multislice\_thick\_sample}: Fluorescent thick sample.
      \item \texttt{PTFT}: Projection tensor for transverse‐wave polarization.
      \item \texttt{thick\_sample\_vector}: Vector field through a thick scattering sample.
    \end{itemize}

  \item[\large\bfseries 9.\ ] sensors.py
    \begin{itemize}[itemsep=0.5em]
      \item \texttt{basic\_sensor}: Produces an intensity image and simulates shot noise.
    \end{itemize}

  \item[\large\bfseries 10.\ ] sources.py
    \begin{itemize}[itemsep=0.5em]
      \item \texttt{point\_source}: Scalar/vectorial point‐source at distance \(z\).
      \item \texttt{objective\_point\_source}: Field after an objective lens (NA, \(f\)).
      \item \texttt{plane\_wave}: Generates a plane wave.
      \item \texttt{generic\_field}: User‑defined phase front for scalar/vectorial fields.
    \end{itemize}
\end{itemize}


\section{Elements Package}
The \textbf{elements} package wraps functions into \texttt{flax.linen.Module} classes, making them trainable modules.
\begin{itemize}

    \item \texttt{AmplitudeMask}: Applies an amplitude mask to an incoming field. The amplitude can be learned pixel by pixel.
    
    \vspace{1em}
    {\large \textbf{1. amplitude\_masks.py}}
    \item \texttt{Polarizer} classes for modeling wave plates, polarizer filters, etc.
    \item \texttt{PhaseMask} and \texttt{AmplitudeMask}: Potentially trainable masks for inverse design.
\end{itemize}

\section{Systems Package}
\begin{itemize}
    \item \texttt{OpticalSystem}: A container that sequentially applies elements, enabling complex system simulation (microscopes, telescopes, etc.).
    \item \texttt{Microscopy} and \texttt{Optical4FSystemPSF}: Pre-built specialized classes for immediate usage.
\end{itemize}

\section{Performance Considerations}
Chromatix performance depends on array shapes and the complexity of transformations. JAX and XLA (Accelerated Linear Algebra) can compile repeated operations for speed. However, high-resolution fields can be memory-intensive, especially if running multiple parallel wave calculations (e.g., multi-wavelength or batch). GPU usage is recommended.

\chapter{Application Examples}
\section{Example 1: Building Optical System A}
Description, purpose, implementation, and results.

\section{Example 2: Building Optical System B}
Description, purpose, implementation, and results.

\chapter{Conclusion and Future Work}
\section{Conclusion}
Summary of your findings, overall impressions, and effectiveness of Chromatix.

\section{Future Work}
Recommendations for further study or improvements to the library and potential applications.

\chapter{Acknowledgments}
I wish to express my sincere gratitude to my supervisor, Prof. Dr. Vladan Blahnik, for supervising my internship. 

Special thanks to Prof. Dr. , retired professor from [University/Institute], CEO of LightTrans GmbH, for providing the internship topic, invaluable insights, and continuous support throughout the project. 

% Include anyone else you want to thank here.


\bibliographystyle{plain}
\bibliography{references}

\appendix
\chapter{Additional Material}
Include additional scripts, code snippets, or supplementary material here.

\end{document}
